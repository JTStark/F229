\documentclass[12pt,a4paper]{article}
\usepackage[utf8]{inputenc}
\usepackage{amsmath}
%\usepackage[brazilian]{babel} % Brazil or not Brazil??
\usepackage{amsfonts}
\usepackage{amssymb}
\usepackage{graphicx}
\usepackage[margin=0.8in]{geometry}


\begin{document}
\title{\vspace{70mm}\Huge Experimento 06a - Calorimetria}
\author{ Giovani Garuffi\qquad\hfill
		\textit {RA: 155559}\protect\\
		João Baraldi\hfill
		\textit{RA: 158044}\protect\\
		Lauro Cruz\hfill
		\textit{RA: 156175}\protect\\
		Lucas Schanner\hfill
		\textit{RA: 156412}\protect\\
		Pedro Stringhini\hfill
		\textit {RA: 156983}								
		}
\maketitle
\newpage
\section{Resumo}


\section{Objetivos}
Este experimento pode ser divido em três partes, cada uma com seus objetivos, que são: traçar um gráfico de calibração de um termopar, calcular a constante de tempo de um calorímetro, e calcular sua capacidade térmica.


\section{Procedimento Experimental e Coleta de Dados}


\subsection{Procedimento}


\subsubsection{Curva de calibração de um termopar}

Nesta parte do experimento preenche-se uma parte do copo do calorímetro (vide figura \ref{calorimetro}) com água fervente e coloca-se nele um termômetro de mercúrio para controle de sua temperatura. Em um béquer separado, coloca-se água com gelo para manter sua temperatura próxima a zero.\\

\begin{figure}[!htbp]
\centering
\includegraphics[scale=0.3]{Fig6a1.jpg}
\caption{Estrutura de um calorímetro.}
\label{calorimetro}
\end{figure}

Então, coloca-se o fio de referência do termopar dentro do copo com gelo e o outro fio dentro do calorímetro(como mostrado na figura \ref{exptermopar}). Então, adiciona-se água à temperatura inicial (retirada da torneira) no calorímetro, e, para cada temperatura da água do calorímetro lê-se uma voltagem (em $mV$) no termopar, anota-se a temperatura e a voltagem, até que primeira se aproxime da inicial.

\begin{figure}[!htbp]
\centering
\includegraphics[scale=0.3]{Fig6a2.jpg}
\caption{Montagem experimental para a calibração do termopar.}
\label{exptermopar}
\end{figure}

Por fim, plota-se um gráfico $V \; x \; T$ com os pares anotados, e um com os tabelados teóricos.\\

\subsubsection{Constante de tempo de um calorímetro}

Como o calorimetro usado não é ideal, nesta parte do experimento, será calculada a sua constante de tempo, que representa a velocidade com que o calorímetro permite trocas de calor com o ambiente.\\
Para tal, prenche-se o calorimetro com água, que deve ser aquecida utilizando-se um ebulidor, e, com ele totalmente fechado (figura \ref{calorimetro}), mas com um termômetro dentro, cronometra-se o tempo com que a temperatura cai. No caso, foi medido o tempo de $T = 89 \; ^{\circ} C$ a $T = 69 \; ^{\circ} C$.\\
Então, pela fórmula 

$$T = T_0 e ^{-t/\tau} + T_a$$

onde $T_0$ é a temperatura inicial, $t$ é o tempo, $\tau$ é a constante a ser encontrada e $T_a$ é a temperatura ambiente (de $26.5 ^{\circ}$, no caso), calcula-se o valor da constante através de um gráfico $ln(T - T_a) \; x \; t$, cuja curva segue a relação semi-logarítmica:

$$T - T_a = T_0 e^{-t/\tau}$$
$$ln(T - T_a) = ln(T_0) - \frac{t}{\tau} ln e$$
$$ln(T - T_a) = - \frac{1}{\tau} \cdot t + ln(T_0)$$

\subsubsection{Capacidade térmica de um calorímetro}

Para determinar a capacidade térmica do calorímetro (figura \ref{calorimetro}), ele teve, aproximadamente, um terço de seu volume preenchido por água à temperatura de $\theta_{frio} = 18 \; ^{\circ}C$. Depois, acrescentou-se a mesma quantidade de água quente a $\theta_{quente} = 82 \; ^{\circ}C$, esperando o sistema entrar em equilibrio e vendo sua temperatura final. Assim, a partir da relação:

$$\Sigma Q = 0$$
$$-Q_{perdido H_2O} = Q_{recebido H_2O} + Q_{recebido Calorimetro}$$
$$-m_{H_2O}\cdot c_{H_2O} \cdot (\theta_{final}-\theta_{quente}) = m_{H_2O}\cdot c_{H_2O} \cdot (\theta_{final}-\theta_{frio}) + C_{cal}\cdot (\theta_{final}-\theta_{frio}) $$ \
$$C_{cal} = \frac{m_{H_2O}\cdot c_{H_2O}\cdot(\theta_{quente} + \theta_{frio} -2\theta_{final})}{(\theta_{final}-\theta_{frio})}$$


\subsection{Dados Obtidos}

A Tabela \ref{dados} apresenta as medições Da tensão medida no termopar, em função da temperatura.

\begin{table}[!htbp]

\centering
\def\arraystretch{1.5}
\caption{Dados obtidos no experimento}

\begin{tabular}{|r|r|}
\hline
Tensão $(mV)$ & Temperatura $(C)$\\
\hline
 4.62 & 89 \\
 \hline
 4.40  & 87 \\
 \hline
 4.19 & 84 \\
 \hline
 3.96 & 80 \\
 \hline
 3.82 & 78 \\
 \hline
 3.04 & 65 \\
 \hline
 2.89 & 62 \\
 \hline
 2.44 & 54 \\
 \hline
 2.31 & 51 \\
 \hline
 2.03 & 47 \\
 \hline
 1.94 & 45 \\
 \hline
 1.80  & 42 \\
 \hline
 1.69 & 40 \\
 \hline
 1.44 & 37 \\
\hline
\end{tabular}

\emph{O erro na temperatura é de $0.5 C$, e na tensão de $0.01 mV$}
\label{dados}
\end{table}

\begin{table}[!htbp]

\centering
\def\arraystretch{1.5}
\caption{Dados obtidos no experimento}

\begin{tabular}{|r|r|}
\hline
Tempo $(s)$ & temperatura $(C)$ \\
\hline
    0 & 89 \\    
  700 & 78 \\
  860 & 76 \\
 1500 & 71 \\
 1610 & 70 \\
\hline
\end{tabular}

\emph{O erro na temperatura é de $0.5 C$, e no tempo de $0.5 s$}
\label{dados}
\end{table}


\section{Análise dos Resultados e Discussões}

\subsection{Curva de Calibração do Termopar}
Para comparar os dados obtidos no experimento e os dados conhecidos de tensão em função da temperatura, foi construído o gráfico na Figura \ref{termopar}.  \\
Verifica-se que houve algum tipo de erro experimental na realização, uma vez que os resultados obtidos são internamente consistentes (A relação é linear, assim como esperado), mas uma diferença significativa das medidas esperadas.

% INSERT migués HERE 
 
\begin{figure}[!htbp]
\includegraphics[scale=0.55]{termopar.png}
\caption{Curva de calibração do termopar. As medidas Azuis são as obtidas experimentalmente e as vermelhas são as esperadas}
\label{termopar}
\end{figure}

Esse deslocamento da reta, foi, provavelmente, resultado erros sistemáticos como o termopar não estar calibrado e/ou a água do béquer de referência não estava necessariamente a $0 \; ^{\circ} C$.

\subsection{Constante de tempo do calorímetro}
A queda de temperatura da agua no calorímetro pode ser descrita pela equação
$$ T = T_0 e^{-t/\tau} + T_a $$
que pode ser reescrita como
$$ \ln \Delta T = -t/\tau + \ln T_0 $$
Vemos então que deve haver uma relação linear entre $ln \Delta T$ e $ t $. Para verificar essa relação foi construido a tabela \ref{table:tempo} e o gráfico da figura \ref{fig:tempo}. \\
\begin{table}[!htbp]
\centering
\def\arraystretch{1.5}
\caption{Dados relacionando $ln \Delta T$ à $t$}
\label{table:tempo}
\begin{tabular}{|c|l|r|r|}
\hline
Temperatura $\; (C)$ & $\Delta T \; (C)$ & $\ln \Delta T \; (\ln C)$ & tempo $\; (s)$ \\
\hline
 $89 \pm 0.5$ & $ 62.5 \pm  0.7 $ & $ 4.135 \pm 0.008 $    &    0 \\
 \hline
 $78 \pm 0.5$ & $ 51.5 \pm  0.7 $ & $ 3.941 \pm 0.009 $ &  700 \\
 \hline
 $76 \pm 0.5 $& $ 49.5 \pm  0.7 $ & $ 3.90 \pm 0.01 $ &  860 \\
 \hline
 $71 \pm 0.5$ & $ 44.5 \pm  0.7 $ & $ 3.79 \pm 0.01 $ & 1500 \\
 \hline
 $70 \pm 0.5$& $ 43.5 \pm  0.7 $ & $ 3.77 \pm 0.01 $ & 1610 \\
\hline
\end{tabular}\\
\emph{$\Delta T$ for calculado a partir de uma temperatura ambiente de $26.5 C$}

\end{table}
Fazendo a regressão linear sobre os dados da tabela \ref{table:tempo}, obtemos os coeficientes 
$$ a = -0.000232 \pm 0.000007\qquad e\qquad b = 4.125 \pm 0.007 $$

\begin{figure}[!htbp]
\includegraphics[scale=0.55]{tempo.png}
\caption{Gŕafico de regressão linear de $ \ln \Delta T$ por $t$.}
\label{fig:tempo}
\end{figure}

Sabemos então que, pela relação $ \ln \Delta T = -t/\tau + \ln T_0 $, $a = -\frac{1}{\tau}$ e, portanto:
$$ \tau = -\frac{1}{a}; $$
$$ \Delta\tau =\frac{\Delta a}{a^2}. $$
Assim, 
$$ \tau = (4300 \pm 100)s $$

\section{Conclusões}


\section{Bibliografia}

\end{document}

