\documentclass[12pt,a4paper]{article}
\usepackage[utf8]{inputenc}
\usepackage{amsmath}
\usepackage[brazilian]{babel} % Brazil or not Brazil??
\usepackage{amsfonts}
\usepackage{amssymb}
\usepackage{graphicx}
\usepackage[margin=0.8in]{geometry}


\begin{document}
\title{\vspace{70mm}\Huge Experimento 06a - Calorimetria}
\author{ Giovani Garuffi\qquad\hfill
		\textit {RA: 155559}\protect\\
		João Baraldi\hfill
		\textit{RA: 158044}\protect\\
		Lauro Cruz\hfill
		\textit{RA: 156175}\protect\\
		Lucas Schanner\hfill
		\textit{RA: 156412}\protect\\
		Pedro Stringhini\hfill
		\textit {RA: 156983}								
		}
\maketitle
\newpage
\section{Resumo}

\section{Objetivos}
Este experimento pode ser divido em duas partes, cada uma com seus objetivos, que são: a determinação do calor específico de três metais diferentes (acreditados de serem chumbo, alumínio e cobre), e a determinação do calor latente de fusão do gelo. 


\section{Procedimento Experimental e Coleta de Dados}


\subsection{Procedimento}


\subsubsection{Determinação do Calor Específico de Metais}

%acentuar palavras 

Esta parte do experimento foi feita da seguinte maneira: com um ebulidor, aquece-se uma amostra de ´agua, numa garrafa t´ermica, e imerge-se a amostra do metal, de massa obtida com uma balança, nessa ´agua, mantendo-se o controle de sua temperatura com um term^ometro de merc´urio. Ent~ao, insere-se o metal aquecido no calor´imetro com ´agua fria (vide figura \ref{CalorMetais}), de temperatura e massa tamb´em conhecidos e espera-se pelo equ´ilibrio t´ermico, para anotar seu valor.

\begin{figure}[!htbp]
\centering
\includegraphics[scale=0.55]{Fig6b1.jpg}
\caption{Exemplo da montagem experimental da primeira parte do experimento.}
\label{CalorMetais}
\end{figure}

Ent~ao, repete-se o procedimento para as demais amostras.

\subsubsection{Determinação do Calor Latente de Fusão do Gelo}

Esta outra parte ´e feita de forma an´aloga. Insere-se uma massa $m$ (encontrado com a balança) de gelo, de temperatura conhecida, no calor´imetro preenchido at´e a metade com agua fria, de massa e temperatura conhcidos, at´e atingir-se o equil´ibrio t´ermico, medindo-se seu valor com o term^ometro, como mostrado na figura \ref{CalorGelo}.

\begin{figure}[!htbp]
\centering
\includegraphics[scale=0.55]{Fig6b2.jpg}
\caption{Exemplo de montagem experimental da segunda parte do experimento.}
\label{CalorGelo}
\end{figure}



\subsubsection{Capacidade térmica de um calorímetro}


\subsection{Dados Obtidos}


\section{Análise dos Resultados e Discussões}


\subsection{Curva de Calibração do Termopar}

\subsection{Constante de tempo do calorímetro}


\section{Conclusões}

\end{document}

