\documentclass[12pt,a4paper]{article}
\usepackage[utf8]{inputenc}
\usepackage{amsmath}
\usepackage[brazilian]{babel} % Brazil or not Brazil??
\usepackage{amsfonts}
\usepackage{amssymb}
\usepackage{graphicx}
\usepackage[margin=0.8in]{geometry}


\begin{document}
\title{\vspace{70mm}\Huge Experimento 04 - Máquina de atwood}
\author{ Giovani Garuffi\qquad\hfill
		\textit {RA: 155559}\protect\\
		João Baraldi\hfill
		\textit{RA: 158044}\protect\\
		Lauro Cruz\hfill
		\textit{RA: 156175}\protect\\
		Lucas Schanner\hfill
		\textit{RA: 156412}\protect\\
		Pedro Stringhini\hfill
		\textit {RA: 156983}								
		}
\maketitle
\newpage
\section{Resumo}
Inicialmente, prendeu-se um fio (inextensível) com duas massas nas extremidades em uma polia em torno de um eixo fixo (Máquina de Atwood). Após variar a diferença entre as massas das extremidades dos pesos dos fios (com discos de metal de massas variadas) e obter os períodos de queda de da massa de maior peso com um cronômetro, foi utilizada a fórmula $Delta m = (\frac{2h}{gR^2})(I + MR^2))(\frac{1}{t^2})+(\frac{tau_a}{gR})$ para determinar o momento de inércia da polia e o torque do atrito.
Após a transformação linear da equação em X, traçou-se um gráfico de Y por Z. A partir desses dados e das dimensões do cilindro (calculadas com um paquímetro), foi possível a determinação do momento de inércia aproximado a partir da fórmula X_2.
% Não me lembro das fórmulas e dos eixos do gráfico, sei q sou várzea, mas me ajudem ;) hahahaha

\section{Objetivos}
O experimento realizado teve como objetivo estudar a máquina de Atwood, utilizando para isso a determinação do momento de inércia da polia utilizada e do torque realizado pelo atrito entre tal polia e o fio que a toca.$T = \sqrt{\frac{8\pi I_0 L}{G r^4}}$


\section{Procedimento Experimental e Coleta de Dados}

\subsection{Materiais utilizados}

\begin{itemize}
	\item Polia de latão com eixo;
	\item barbante;
	\item dois pesos de suspensão;
	\item vários discos metálicos que se acoplam aos pesos;
	\item fita métrica para medição de comprimentos;
	\item paquímetro;
	\item balança de precisão;
	\item cronômetro;
\end{itemize}

\subsection{Procedimento}





\subsection{Dados Obtidos}


\section{Análise dos Resultados e Discussões}
\subsection{Linearização}


\subsection{Regressão linear}

\subsection{Significado do coeficiente angular}

\section{Conclusões}




\end{document}

