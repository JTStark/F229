\documentclass[12pt,a4paper]{article}
\usepackage[utf8]{inputenc}
\usepackage{amsmath}
%\usepackage[brazilian]{babel} % Brazil or not Brazil??
\usepackage{amsfonts}
\usepackage{amssymb}
\usepackage{graphicx}
\usepackage[margin=0.8in]{geometry}


\begin{document}
\title{\vspace{70mm}\Huge Experimento 04 - Máquina de atwood}
\author{ Giovani Garuffi\qquad\hfill
		\textit {RA: 155559}\protect\\
		João Baraldi\hfill
		\textit{RA: 158044}\protect\\
		Lauro Cruz\hfill
		\textit{RA: 156175}\protect\\
		Lucas Schanner\hfill
		\textit{RA: 156412}\protect\\
		Pedro Stringhini\hfill
		\textit {RA: 156983}								
		}
\maketitle
\newpage
\section{Resumo}

\section{Objetivos}
O experimento realizado teve como objetivo estudar a máquina de Atwood, utilizando para isso a determinação do momento de inércia da polia utilizada e do torque realizado pelo atrito entre tal polia e o fio que a toca.


\section{Procedimento Experimental e Coleta de Dados}

\subsection{Materiais utilizados}
Para a realização deste experimento, foram utilizados os seguintes materiais:

\begin{itemize}
	\item Polia de latão com eixo;
	\item barbante;
	\item dois pesos de suspensão;
	\item vários discos metálicos que se acoplam aos pesos;
	\item fita métrica para medição de comprimentos;
	\item paquímetro;
	\item balança de precisão;
	\item cronômetro.
\end{itemize}

\subsection{Procedimento}
	O experimento da Máquina de Atwood consiste em dois pesos de suspensão ligados por um fio leve e inextensível (foi utilizado um pedaço de barabante), que passa por uma polia, um cilindro (no caso, um de latão de raio $R$ e momento de inércia $I$), como mostra a figura 1. \\

Entre os pesos de suspensão, são distribuídos discos metálicos que aumentam sua massa total (vide figura 2). O objetivo desses discos é variar a massa de cada peso, mas manter a soma dos dois constantes. Para tal, basta-se apenas passar os discos de um peso para o outro, para assim, dada a equação

$$\Delta m = (2h/gR^2)(I + MR^2)(1/t^2) + \tau_a/(gR),$$

onde $\Delta m$ é a diferença entre as massas dos pesos com os discos e $M$ é a soma delas, apenas o primeiro varie, mas o segundo matenha-se constante.\\

Feito isso, para cada valor de $\Delta m$, o experimento, que consiste em abandonar o corpo mais pesado da altura $h$ e medir o tempo $t$ em que isso ocorre, foi realizado três vezes.

\subsection{Dados Obtidos}


\section{Análise dos Resultados e Discussões}
\subsection{Linearização}


\subsection{Regressão linear}

\subsection{Significado do coeficiente angular}

\section{Conclusões}




\end{document}

